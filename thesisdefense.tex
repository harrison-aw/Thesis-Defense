\documentclass{beamer}
\usetheme{Berlin}
\usecolortheme{seahorse}

\title{Bounding the order of a group with a large conjugacy class}
\author{Anthony Harrison}
\institute{Texas State University-San Marcos}

\begin{document}

\begin{frame}
\titlepage
\end{frame}

\section{Introduction}

\begin{frame}
\frametitle{Introduction}
About this talk:
\begin{itemize}
	\item We will discuss a topic in abstract algebra.
	\item Specifically, it will deal with finite groups.
	\item More specifically, it is the analog to a problem in character theory.
\end{itemize}
\end{frame}

\begin{frame}
\frametitle{Notation and Terms}
\end{frame}

\begin{frame}
\frametitle{Groups with a character of large degree}
Noah Snyder introduced a parameter $e$ that is ``small'' when a group has an irreducible character with ``large'' degree.
\vskip1cm
He showed it was possible to bound the order of a group by $e$: $|G| \le ((2e)!)^2$.
\end{frame}

\begin{frame}
\frametitle{Bounding the order of a group with a large character degree}
A lot of research has gone into this bound:
\begin{itemize}
	\item Martin Isaacs: $|G| \le Be^6$ for some universal constant $B$.
	\item Christina Durfee and Sara Jensen: $|G| \le e^6 - e^4$.
	\item Mark Lewis: $|G| \le e^4 - e^3$ if $G$ has a nontrivial, abelian normal subgroup.
\end{itemize}
\end{frame}



\begin{frame}
\frametitle{Group Theoretic Analog}
There is a close relationship between irreducible characters and conjugacy classes
\begin{itemize}
	\item Characters are constant functions over a given conjugacy class.
	\item The number of conjugacy classes equals the number irreducible characters.
\end{itemize}
We will redefine $e$ to use conjugacy classes instead of irreducible character degrees.
\end{frame}


\section{Background}

\begin{frame}
\frametitle{Group Actions}
Let $G$ be a group and $\Omega$ be a set. A \emph{group action} is a homomorphism $\pi: G \to Sym(\Omega)$, where $Sym(\Omega)$ is the symmetric group of $\Omega$.
\begin{itemize}
	\item<2-> The image of $G$ under $\pi$ is called the \emph{permutation representation} of $G$.
\end{itemize}
\end{frame}

\begin{frame}
\frametitle{Orbits}
\end{frame}

\begin{frame}
\frametitle{Conjugacy Classes}
\end{frame}

\begin{frame}
\frametitle{Permutation Representations}
\end{frame}

\begin{frame}
\frametitle{Dihedral Groups}
\end{frame}

\begin{frame}
\frametitle{Generalized Dihedral Groups}
\end{frame}


\section{Main Result}

\begin{frame}
\frametitle{The Parameter}
\end{frame}

\begin{frame}
\frametitle{Examples}
\end{frame}

\begin{frame}
\frametitle{Lemma 1}
\end{frame}

\begin{frame}
\frametitle{Lemma 2}
\end{frame}

\begin{frame}
\frametitle{Lemma 3}
\end{frame}

\begin{frame}
\frametitle{Theorem 4}
\end{frame}

\section{What's Next?}

\begin{frame}
\frametitle{Relative e}
\end{frame}

\begin{frame}
\frametitle{References}
\end{frame}

\end{document}