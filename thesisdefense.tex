\documentclass{beamer}
\usetheme{Berlin}
\usecolortheme{seahorse}

\usepackage{amsmath}

\title{Bounding the order of a group with a large conjugacy class}
\author{Anthony Harrison}
\institute{Texas State University-San Marcos}

\begin{document}

\begin{frame}
\titlepage
\end{frame}

\section{Introduction}

\begin{frame}
\frametitle{Introduction}
About this talk:
\begin{itemize}
	\item We will discuss a topic in abstract algebra.
	\item Specifically, it will deal with finite groups.
	\item More specifically, it is the analog to a problem in character theory.
\end{itemize}
\end{frame}

\begin{frame}
\frametitle{Notation and Terms}
\end{frame}

\begin{frame}
\frametitle{Groups with a character of large degree}
Noah Snyder introduced a parameter $e$ that is ``small'' when a group has an irreducible character with ``large'' degree.
\begin{itemize}
	\item<1-> $|G| = d(d+e)$ where $d$ is the degree of an ordinary irreducible character of $G$.
	\item<2-> $e$ is always an integer.
	\item<3-> $|G| \le ((2e)!)^2$.
\end{itemize}
\uncover<4->{
\vskip0.5cm
The bound on the order of $G$ inspired at least 3 papers.
}
\end{frame}

\begin{frame}
\frametitle{Bounding the order of a group with a large character degree}
\begin{itemize}
	\item Martin Isaacs: $|G| \le Be^6$ for some universal constant $B$.
	\item Christina Durfee and Sara Jensen: $|G| \le e^6 - e^4$.
	\item Mark Lewis: $|G| \le e^4 - e^3$ if $G$ has a nontrivial, abelian normal subgroup.
\end{itemize}
\end{frame}



\begin{frame}
\frametitle{Group Theoretic Analog}
There is a close relationship between irreducible characters and conjugacy classes.
\begin{itemize}
	\item Characters are constant functions over a given conjugacy class.
	\item The number of conjugacy classes equals the number irreducible characters.
\end{itemize}
\pause
We will redefine $e$ to use conjugacy class sizes instead of irreducible character degrees.
\end{frame}

\begin{frame}
\frametitle{Main Result}
Our goal will be to:
\begin{itemize}
	\item Bound the order of a group with the newly defined $e$.
	\item Classify all groups attaining this bound.
\end{itemize}
\end{frame}


\section{Background}

\begin{frame}
\frametitle{Group Actions}
\begin{block}{Definition}
Let $G$ be a group and $\Omega$ be a set. A \emph{group action} is a homomorphism $\pi: G \to Sym(\Omega)$, where $Sym(\Omega)$ is the symmetric group of $\Omega$.
\end{block}
\begin{itemize}
	\item<2-> The image of $G$ under $\pi$ is called the \emph{permutation representation} of $G$.
	\item<3-> We will denote the bijection $\pi(g)$ for $g \in G$ by $\pi_g$.
\end{itemize}
\end{frame}

\begin{frame}
\frametitle{Examples of Group Actions}
Let $G = S_n$ and $\Omega = \{1, 2, \ldots, n\}$. Then $\pi$ is an isomorphism and $\pi_g$ for $g \in S_n$ simply permutes elements of $\Omega$.
\vskip0.5cm
\uncover<2->{
Let $G$ be a group and $\Omega = G$. Define $\pi: G \to Sym(G)$ to be the homomorphism such that $\pi_g(h)= g^{-1}hg$ for $h, g \in G$. This is the conjugation action of $G$ on itself.
}
\end{frame}

\begin{frame}
\frametitle{Orbits}
\begin{block}{Definition}
The \emph{orbit} of $\alpha \in \Omega$ is the set $\mathcal{O}_\alpha = \{ \pi_g(\alpha) : g \in G \}$.
\end{block}
\vskip0.5cm
\uncover<2->{
Let $S_n$ act on $\Omega = \{1,2,\ldots,n\}$. This action has only a single orbit, $\mathcal{O}_1 = \Omega$.
}
\vskip0.5cm
\uncover<3->{
Let $G$ be an abelian group and let it act on itself by conjugation. Every orbit is a singleton: $\mathcal{O}_g = \{ g \}$ for all $g \in G$.
}
\end{frame}

\begin{frame}
\frametitle{Stabilizers}
\begin{block}{Definition}
The \emph{stabilizer} of $\alpha \in \Omega$ is the group $G_\alpha = \{ g \in G : \pi_g(\alpha) = \alpha \}$.
\end{block}
\vskip0.5cm
\uncover<2->{
Let $S_n$ act on $\Omega = \{1,2,\ldots,n\}$. The stabilizer of each $g \in S_n$ has order $(n-1)!$.
}
\vskip0.5cm
\uncover<3->{
Let $G$ act on itself by conjugation. The stabilizer of $g \in G$ is $G_g = C_G(g)$, the centralizer of $g$ in $G$.
}
\end{frame}

\begin{frame}
\frametitle{Orbit-Stabilizer Theorem}
\begin{block}{Theorem}
Let $G$ be a group acting on a set $\Omega$. For $\alpha \in \Omega$, $|\mathcal{O}_\alpha| = |G : G_\alpha|$.
\end{block}
\vskip0.5cm
\uncover<2->{
Let $S_n$ act on $\Omega = \{1,2,\ldots,n\}$. For $g \in S_n$, $|\mathcal{O}_g| = n! / (n-1)! = n$.
}
\vskip0.5cm
\uncover<3->{
Let $G$ be a group acting on itself by conjugation. For $g \in G$, $|\mathcal{O}_g| = |G : C_G(g)|$.
}
\end{frame}

\begin{frame}
\frametitle{Conjugacy Classes}
\begin{block}{Definition}
Let $G$ be a group. The \emph{conjugacy class} of $g \in G$ is the orbit of the $g$ under the conjugation action of $G$ on itself.
\end{block}
\begin{itemize}
	\item The size of the conjugacy class of $g \in G$ is $|G : C_G(g)|$.
\end{itemize}
\end{frame}

\begin{frame}
\frametitle{Dihedral Groups}
\begin{block}{Definition}
A dihedral group is a group $G$ with a cyclic subgroup $C$ of index 2 such that $G - C$ contains only involutions.
\end{block}
\begin{itemize}
	\item If $C$ has odd order, $G$ has exactly one conjugacy class of involutions.
	\item If $C$ has even order, $G$ has exactly three conjugacy classes of involutions.
\end{itemize}
\end{frame}

\begin{frame}
\frametitle{Generalized Dihedral Groups}
\begin{block}{Definition}
A \emph{generalized dihedral group} is a group $G$ with an abelian subgroup $B$ of index 2 such that $G - B$ contains only involutions.
\end{block}
\begin{itemize}
	\item If $B$ has odd order, $G$ has exactly one class of involutions.
\end{itemize}
\end{frame}


\section{Main Result}

\begin{frame}
\frametitle{Defining e}
\begin{block}{Definition}
Let $G$ be a group and $d$ be the square root of its largest conjugacy class size. Define $e$ via the equation $|G| = d(d + e)$.
\end{block}
\begin{itemize}
	\item $e$ is not an integer!
	\item $e = (|C_G(x)| - 1) \cdot \sqrt{|G : C_G(x)|}$ where $x$ is a member of the largest conjugacy class of $G$.
\end{itemize}
\end{frame}

\begin{frame}
\frametitle{Calculating e}
\end{frame}

\begin{frame}
\frametitle{Lemma 1}
\end{frame}

\begin{frame}
\frametitle{Lemma 2}
\end{frame}

\begin{frame}
\frametitle{Lemma 3}
\end{frame}

\begin{frame}
\frametitle{Theorem 4}
\end{frame}

\section{What's Next?}

\begin{frame}
\frametitle{Relative e}
\end{frame}

\begin{frame}
\frametitle{References}
\end{frame}

\end{document}